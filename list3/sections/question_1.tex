% !TeX spellcheck = en_US
%question 1
\section{Question 1}
A control system with state $\mathbf{x}$ is described by matrices:
\begin{equation*}
\mathbf{A} =
\begin{bmatrix}
-2 & 1 \\ -2 & 0
\end{bmatrix},
\mathbf{B} =
\begin{bmatrix}
1 \\ 3
\end{bmatrix},
\mathbf{C} = 
\begin{bmatrix}
1 & 0
\end{bmatrix},
\mathbf{D} = 
\begin{bmatrix}
0
\end{bmatrix}
\end{equation*}

\paragraph{(a)} The transfer function ($G_{(s)}$) can be computed as
\begin{align}
	G_{(s)}&=\mathbf{C}(\mathbf{I}s-\mathbf{A})^{-1}\mathbf{B}, \nonumber \\
	G_{(s)}&= \frac{s+3}{s^2 + 2s + 2},
	\label{eq:1a_G}
\end{align}

\paragraph{(b)} Canonical form of system using \eqref{eq:1a_G} is given by
\begin{equation*}
\mathbf{A_c} =
\begin{bmatrix}
-2 & -2 \\ 1 & 0
\end{bmatrix},
\mathbf{B_c} =
\begin{bmatrix}
1 \\ 0
\end{bmatrix},
\mathbf{C_c} = 
\begin{bmatrix}
1 & 3
\end{bmatrix},
\mathbf{D_c} = 
\begin{bmatrix}
0
\end{bmatrix}.
\end{equation*}

Considering $\mathbf{z}$ as states of canonical form and $\mathbf{z}=\mathbf{T}\mathbf{x}$. Then, matrix $\mathbf{T}$ can be computed solving $\mathbf{A T} = \mathbf{T A_c}$ and $\mathbf{b} = \mathbf{T b_c}$:
\begin{equation*}
\mathbf{T} = 
\begin{bmatrix}
	-0.08 & 0.36 \\ 0.36 & 0.88
\end{bmatrix}.
\end{equation*}
 








